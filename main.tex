%xelatex 或 pdflatex 编译
%导言区
\documentclass[UTF8]{article} %UTF8编码
%preface for the main.tex
\usepackage[a4paper]{geometry} %设置纸张为A4大小
\usepackage{amsmath} %最常用的数学宏包
\usepackage{amssymb} %提供了额外的数学字体和数学符号
\usepackage{mathtools} %提供了额外的数学工具,如dcases环境
\usepackage{lipsum} %英语假文宏包
\usepackage{graphicx} %插入图片宏包
\usepackage{float} %在浮动体中可使用H选项
\usepackage{extarrows} %提供了额外的箭头,如可随文字延长的各种箭头
\usepackage{array} %可增加更多的表格列限定符
\usepackage{booktabs} %可使用三线表
\usepackage{bm} %可在数学环境中使用斜体加粗的命令
\usepackage[dvipsnames]{xcolor} %扩展版的color宏包
\usepackage{ctex} %允许在文中使用中文

\usepackage{hyperref} %可使用超链接(含链接的跳转和一些命令)
\hypersetup{
	colorlinks=true,
	citecolor=magenta,%设置cite类命令超链接的颜色
	linkcolor=blue,%设置目录、脚注、ref等超链接的颜色
	urlcolor=violet,%设置网页超链接的颜色
}

\usepackage[amsmath,thmmarks]{ntheorem} %定理类环境宏包,如果前面使用amsmath宏包,则需加上amsmath宏包选项以避免出现未知问题,若需在定理环境末尾加上特定符号(如证毕符号),则需使用thmmarks宏包选项以使用\theoremsymbol{}命令。
{
	\theoremstyle{plain}
	\setlength{\theoremindent}{0em}
	\newtheorem{definition}{Definition}
}
{
	\theoremstyle{plain}
	\setlength{\theoremindent}{0em}
	\newtheorem{lemma}[definition]{Lemma}
}
{
	\theoremstyle{plain}
	\setlength{\theoremindent}{0em}
	\newtheorem{theorem}[definition]{Theorem}
}
{
	\theoremstyle{plain}
	\setlength{\theoremindent}{0em}
	\newtheorem{corollary}[definition]{Corollary}
}
{
	\theoremstyle{nonumberplain}
	\setlength{\theoremindent}{0em}
	\theorembodyfont{\normalfont}
	\theoremsymbol{$\blacksquare$} %在证明环境末尾加上一个证毕符号
	\newtheorem{proof}{Proof}
}

%定义新命令与新运算符
\DeclareMathOperator{\diff}{d\!}
\newcommand{\var}{\mathrm{Var}}
\newcommand{\cov}{\mathrm{Cov}}

%自定义列表环境
\renewcommand{\theenumii}{\arabic{enumii}}

%标题页设置
\title{
	第一次高微作业
}
\author{王美庭\thanks{王美庭, 学号:202020111002, Email: wangmeiting92@gmail.com}}
\date{\today}




%正文区
\begin{document}
\maketitle

\begin{enumerate}
    %problem boundary--------------------------------------------------------------------
    \item 以单一物品的拍卖问题为例(一个卖者出售一种不可分单一物品,有 $N\geq 2$ 个潜在买者),证明以下几个引理:
    \begin{enumerate}
        \item 如果一个直接机制是激励相容的,那么对于每一个行为人 $i \in I,$ 函数 $Q_{i}$ 是递增的。
        
        \textbf{答:}设有 $i,j\in I$,其真实特征分别为 $\theta_i,\theta_j \in [\underline{\theta},\overline{\theta}]$(不妨设 $\theta_i > \theta_j$),获得商品概率的期望分别为 $Q_i(\theta_i), Q_j(\theta_j)$, 需要支付的期望值分别 $T_i(\theta_i), T_j(\theta_j)$。如果一个直接机制是激励相容的,则有
        \begin{gather*}
            \theta_i Q_i(\theta_i) - T_i(\theta_i) \geq \theta_i Q_i(\theta_j) - T_i(\theta_j)  \\
            \theta_j Q_j(\theta_i) - T_j(\theta_i) \leq \theta_j Q_j(\theta_j) - T_j(\theta_j) 
        \end{gather*}
        由于对于每一个 $i\in I$,$\theta_i$ 为独立同分布,则有 $Q_j(\theta_i)=Q_i(\theta_i)$, $T_j(\theta_i)=T_i(\theta_i)$。将该思想应用于以上方程组,
        \begin{gather}
            \theta_i Q_i(\theta_i) - T_i(\theta_i) \geq \theta_i Q_i(\theta_j) - T_i(\theta_j) \label{eq:1-1-1} \\
            \theta_j Q_i(\theta_i) - T_i(\theta_i) \leq \theta_j Q_i(\theta_j) - T_i(\theta_j) \label{eq:1-1-2}
        \end{gather}
        方程 \eqref{eq:1-1-1} - \eqref{eq:1-1-2}, 
        \begin{gather*}
            (\theta_i-\theta_j)Q_i(\theta_i) \geq (\theta_i-\theta_j)Q_i(\theta_j) \implies Q_i(\theta_i) \geq Q_i(\theta_j)
        \end{gather*}
        可得函数 $Q_i$ 是递增的。
        
        \item 如果一个直接机制是激励相容的, $U_{i}$ 是递增的,且对于使得 $U_{i}$ 可微的所有 $\theta_i$, $U_{i}$ 满足 $U_{i}^{\prime}\left(\theta_{i}\right)=Q_{i}(\theta_i)$。
        
        \textbf{答:}如果一个直接机制是激励相容的, 则有
        \[ U_i(\theta_i) = \theta_i Q_i(\theta_i) - T_i(\theta_i) \]
        在 (1) 中已经提到,函数 $Q_i(x)$, $T_i(x)$ 不受下标 $i$ 的影响,因此函数 $U_i(x)$ 也不受下标 $i$ 的影响,则在不失一般性的情况下,这里在计算过程中将略去相关字母的下标 $i$。令 $\delta>0$,对于 $U(\theta)$ 的右导数(这里假设 $U(\theta)$ 的导数存在),
        \begin{gather}
            \begin{split}
                \lim_{\delta\to 0} \frac{U(\theta+\delta) - U(\theta)}{\delta} &= \lim_{\delta\to 0} \frac{ \left[ (\theta+\delta)Q(\theta+\delta) - T(\theta+\delta) \right] -\left[ \theta Q(\theta) - T(\theta) \right]}{\delta} \\
                &\geq \lim_{\delta\to 0} \frac{ \left[ (\theta+\delta)Q(\theta) - T(\theta) \right] -\left[ \theta Q(\theta) - T(\theta) \right]}{\delta} = Q(\theta)
            \end{split} \label{eq:1-2-1}
        \end{gather}
        对于 $U(\theta)$ 的左导数,
        \begin{gather}
            \begin{split}
                \lim_{\delta\to 0} \frac{U(\theta) - U(\theta-\delta)}{\delta} &= \lim_{\delta\to 0} \frac{ \left[ \theta Q(\theta) - T(\theta) \right] - \left[ (\theta-\delta)Q(\theta-\delta) - T(\theta-\delta) \right]}{\delta} \\
                &\leq \lim_{\delta\to 0} \frac{ \left[ \theta Q(\theta) - T(\theta) \right] - \left[ (\theta-\delta)Q(\theta) - T(\theta) \right]}{\delta} = Q(\theta)
            \end{split} \label{eq:1-2-2}
        \end{gather}
        由于 $U(\theta)$ 的导数存在,结合方程 \eqref{eq:1-2-1} 和 \eqref{eq:1-2-2},易得 $U^\prime(\theta) = Q(\theta)$,即 $U_i^\prime(\theta_i) = Q_i(\theta_i)$。
        
        \item $U_{i}\left(\theta_{i}\right)=U_{i}(\underline{\theta})+\int_{\underline{\theta}}^{\theta_{i}} Q_{i}(x) \diff x$
        
        \textbf{答:}由牛顿-莱布尼茨公式,有
        \[ U_{i}\left(\theta_{i}\right) - U_{i}(\underline{\theta}) = \int_{\underline{\theta}}^{\theta_{i}} U^\prime_{i}(x) \diff x = \int_{\underline{\theta}}^{\theta_{i}} Q_{i}(x) \diff x \]
        即
        \begin{gather}
        U_{i}\left(\theta_{i}\right)=U_{i}(\underline{\theta})+\int_{\underline{\theta}}^{\theta_{i}} Q_{i}(x) \diff x \label{eq:1-3-1}
        \end{gather}
        
        \item $T_{i}\left(\theta_{i}\right)=T_{i}(\underline{\theta})+\left[\theta_{i} Q_{i}\left(\theta_{i}\right)-\underline{\theta} Q_{i}(\underline{\theta})\right]-\int_{\underline{\theta}}^{\theta_{i}} Q_{i}(x) \diff x$
        
        \textbf{答:}由于
        \begin{gather*}
            U_i(\theta_i) = \theta_i Q_i(\theta_i) - T_i(\theta_i) \\
            U_i(\underline{\theta}) = \underline{\theta} Q_i(\underline{\theta}) - T_i(\underline{\theta})
        \end{gather*}
        代入方程 \eqref{eq:1-3-1},
        \[ \theta_i Q_i(\theta_i) - T_i(\theta_i) = \underline{\theta} Q_i(\underline{\theta}) - T_i(\underline{\theta}) + \int_{\underline{\theta}}^{\theta_{i}} Q_{i}(x) \diff x \]
        整理得
        \begin{gather}
            T_{i}\left(\theta_{i}\right)=T_{i}(\underline{\theta})+\left[\theta_{i} Q_{i}\left(\theta_{i}\right)-\underline{\theta} Q_{i}(\underline{\theta})\right]-\int_{\underline{\theta}}^{\theta_{i}} Q_{i}(x) \diff x
        \end{gather}
    \end{enumerate}
    
    
    %problem boundary--------------------------------------------------------------------
    \item $c=1$, $v(q)=\ln q$, $\theta$ 在 $[0,1]$ 上服从均匀分布,即对于所有的 $\theta\in[0,1]$, $F(\theta)=\theta$ 且 $f(\theta)=1$。求解垄断卖者的最优定价策略。
    
    \textbf{答:}由于
    \[ \theta - \frac{1-F(\theta)}{f(\theta)} = \theta - \frac{1-\theta}{1} = 2\theta - 1 \]
    关于 $\theta$ 递增,所以 $F$ 是正则分布。在此条件下,对 $q$ 的期望利润最大化选择将由下列条件给出:
    \begin{itemize}
        \item 如果
        \[ v^\prime(0)\left[ \theta - \frac{1-F(\theta)}{f(\theta)} \right] \leq c \]
        那么 $q(\theta) = 0$。
        \item 否则,
        \[ v^\prime\left( q(\theta) \right)\left[ \theta - \frac{1-F(\theta)}{f(\theta)} \right] = c \]
        此时所对应的 $t(\theta)$ 为
        \[ t(\theta) = \theta v\left( q(\theta) \right) - \int_{\underline{\theta}}^\theta v\left( q(x) \right) \diff x \]
    \end{itemize}
    在第一个条件下,我们可以确认对于 $\theta$ 的哪些值,最优供给量 $q(\theta) = 0$:
    \begin{align*}
        v^\prime(0)\left[ \theta - \frac{1-F(\theta)}{f(\theta)} \right] &\leq c \iff \\
        \theta - \frac{1-F(\theta)}{f(\theta)} &\leq 0 \iff \\
        2\theta - 1 &\leq 0 \iff \\
        \theta &\leq 0.5
    \end{align*}
    如果 $\theta > 0.5$,我们将进入上述第二个条件,此时最优供给量 $q(\theta)$ 将由下式给出:
    \begin{align*}
        v^\prime\left( q(\theta) \right)\left[ \theta - \frac{1-F(\theta)}{f(\theta)} \right] &= c \iff \\
        \frac{1}{q}(2\theta-1) &= 1 \iff \\
        q &= 2\theta-1
    \end{align*}
    对于货币转移支付 $t(\theta)$,如果 $\theta\leq 0.5$,则其为 0;如果 $\theta>0.5$,则
    \[ t(\theta) = \theta v\left( q(\theta) \right) - \int_{\underline{\theta}}^\theta v\left( q(x) \right) \diff x = \theta\ln(2\theta-1) - \int_{0.5}^\theta \ln(2x-1) \diff x \]
    其中
    \begin{align*}
       \int_{0.5}^\theta \ln(2x-1) \diff x &\xlongequal{\text{令}\ 2x-1 = t} \frac{1}{2} \int_0^{2\theta-1} \ln t \diff t \\
       &= \frac{1}{2}\left( t\ln t \Big|_0^{2\theta-1} - \int_0^{2\theta-1} t \diff\,\ln t \right) \\
       &= \frac{1}{2} \left[ (2\theta-1)\ln(2\theta-1) - (2\theta-1) \right] \\
       &= \left(\theta-\frac{1}{2}\right)\ln(2\theta-1) - \left(\theta-\frac{1}{2}\right)
    \end{align*}
    所以
    \begin{align*}
        t(\theta) &= \theta\ln(2\theta-1) - \left[ \left(\theta-\frac{1}{2}\right)\ln(2\theta-1) - \left(\theta-\frac{1}{2}\right) \right] \\
        &= \frac{1}{2}\ln(2\theta-1) + \left(\theta-\frac{1}{2}\right)
    \end{align*}
    将 $\theta$ 与 $q$ 关系 ($q=2\theta-1$) 带入 $t(\theta)$, 得
    \[ t(q) = \frac{1}{2}\ln q + \frac{1}{2}q,\ 0 < q \leq 1 \]
    由于垄断卖者不可能接受负的支付,所以我们要求 $t(q)$ 非负。由于 $t(q)$ 在 $(0,1]$ 上递增,且
    \[ \lim_{q\to 0} t(q) = -\infty,\ t(1)=\frac{1}{2} > 0 \]
    所以必存在 $q^* \in (0,1)$,使得 $t(q^*) = 0$,且当 $q\in (q^*,1]$ 时,有 $t(q)>0$。综上,垄断卖者的最优定价策略为:买者可以购买任意数量 $q\in [q^*,1]$ 的商品,然后必须付出 $t(q)=\frac{1}{2}\ln q + \frac{1}{2}q$ 的成本。
\end{enumerate}
\end{document}
