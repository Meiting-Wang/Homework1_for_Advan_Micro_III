%preface for the main.tex
\usepackage[a4paper]{geometry} %设置纸张为A4大小
\usepackage{amsmath} %最常用的数学宏包
\usepackage{amssymb} %提供了额外的数学字体和数学符号
\usepackage{mathtools} %提供了额外的数学工具,如dcases环境
\usepackage{lipsum} %英语假文宏包
\usepackage{graphicx} %插入图片宏包
\usepackage{float} %在浮动体中可使用H选项
\usepackage{extarrows} %提供了额外的箭头,如可随文字延长的各种箭头
\usepackage{array} %可增加更多的表格列限定符
\usepackage{booktabs} %可使用三线表
\usepackage{bm} %可在数学环境中使用斜体加粗的命令
\usepackage[dvipsnames]{xcolor} %扩展版的color宏包
\usepackage{ctex} %允许在文中使用中文

\usepackage{hyperref} %可使用超链接(含链接的跳转和一些命令)
\hypersetup{
	colorlinks=true,
	citecolor=magenta,%设置cite类命令超链接的颜色
	linkcolor=blue,%设置目录、脚注、ref等超链接的颜色
	urlcolor=violet,%设置网页超链接的颜色
}

\usepackage[amsmath,thmmarks]{ntheorem} %定理类环境宏包,如果前面使用amsmath宏包,则需加上amsmath宏包选项以避免出现未知问题,若需在定理环境末尾加上特定符号(如证毕符号),则需使用thmmarks宏包选项以使用\theoremsymbol{}命令。
{
	\theoremstyle{plain}
	\setlength{\theoremindent}{0em}
	\newtheorem{definition}{Definition}
}
{
	\theoremstyle{plain}
	\setlength{\theoremindent}{0em}
	\newtheorem{lemma}[definition]{Lemma}
}
{
	\theoremstyle{plain}
	\setlength{\theoremindent}{0em}
	\newtheorem{theorem}[definition]{Theorem}
}
{
	\theoremstyle{plain}
	\setlength{\theoremindent}{0em}
	\newtheorem{corollary}[definition]{Corollary}
}
{
	\theoremstyle{nonumberplain}
	\setlength{\theoremindent}{0em}
	\theorembodyfont{\normalfont}
	\theoremsymbol{$\blacksquare$} %在证明环境末尾加上一个证毕符号
	\newtheorem{proof}{Proof}
}

%定义新命令与新运算符
\DeclareMathOperator{\diff}{d\!}
\newcommand{\var}{\mathrm{Var}}
\newcommand{\cov}{\mathrm{Cov}}

%自定义列表环境
\renewcommand{\theenumii}{\arabic{enumii}}